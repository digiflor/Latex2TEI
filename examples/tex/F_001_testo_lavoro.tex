
\begin{document}


\begin{estrofa}

\numero{1} \textparagraph\ Axi com cell / quen lo somnis delita\footnote{Aquesta poesia no
es troba origin\`{a}riament al ms. F. Est\`{a} transcrita per una m\`{a}
diferent, del segle XVI que ha copiat el text possiblement de l'edici\'{o} de
Roman\'{\i}. Observem coincid\`{e}ncies amb aquesta edici\'{o} en el vers 29
(llanc en comptes de llas) i en el fet que no porta tornada.}

\numero{2} e son delit / de foll pensament ve

\numero{3} ne pren a mi / quel temps passat me te

\numero{4} limaginar / que altre be\textit{n} noy habita

\numero{5} sentint venir / la guait de ma dolor

\numero{6} sabent de cert / quen sens mans he de jaure

\numero{7} temps por venir / en ningun bem pot caure

\numero{8} so q\textit{ue} no es res / en mi es lo millor /.

\end{estrofa}



\begin{estrofa}

\numero{9} Al temps passat / me trobe gran amor

\numero{10} Amant no res / puis tot es ja finit

\numero{11} daquest pensar / me so jorn em delit

\numero{12} mas quant lo pert / sesforsa ma dolor

\numero{13} si com aquel / qui es jutjat a mort

\numero{14} si de lonch temps / la sap i sa conorta

\numero{15} si creur$\vert{}$el fan / li sera estorta

\numero{16} el fan morir sens vn punt de recort /.

\end{estrofa}



\begin{estrofa}

\numero{17} Plagues a deu / q\textit{ue} mon pensar fos mort

\numero{18} que passas / ma vida en durment

\numero{19} malament viu / qui ten son pensament

\numero{20} per enemich / fentli de mals ha port

\numero{21} que com lo vol / de algun plaer seruir

\numero{22} prenlin axi / com dona ab son infant

\numero{23} que si veri / li demana plorant

\numero{24} ha tant poch seni / que nol sap contradir /.

\end{estrofa}



\begin{estrofa}

\numero{25} [guarda v]fora millor / ma dolor soferir

\numero{26} q\textit{ue} no mesclar poca part de plaher

\numero{27} entro aqueles males / quem gitem de saber

\numero{28} com del plaher passat / me conve exir

\numero{29} llanc mon delit / dolor se converteix

\numero{30} dobles la fam apres dun poch repos

\numero{31} si col malalt / que per un pla\interlin{h}ent mos

\numero{32} tot son menjar en dolor se nodreix

\end{estrofa}



\begin{estrofa}

\numero{33} Com lermirta / que enyorament nol creix

\numero{34} daquelles amichs / q\textit{ue} auia en lo mon

\numero{35} tant ha llonch temps q\textit{ue} en lo poblat no fon

\numero{36} per cas fortuit / hu\textit{n} dels li apareix

\numero{37} qui los plahers / passats li renouella

\numero{38} si quel passat / present li fa tornar

\numero{39} mas com sens part / les forcat congoxar

\numero{40} lo ben com fuig / ab grans crits malapella /.

\end{estrofa}



\end{document}